\title{How we approach}
\author{
}
\date{\today}

\documentclass[12pt]{article}

\begin{document}
\maketitle

\begin{enumerate}
	\item Make the question more formal to avoid ambiguity while communications.
		\begin{enumerate}
			\item define symbols and define all the things. 
				(so that if any other group members also work on the same problem with unique notations ( for a question)).
			\item while defining tray to be as much as generalised i.e. (in a question a ball in rolling down the inclined you should have to consider the air resistance and the resistance is not uniform that also make torque on the ball due to that contact force also change like that)
				\\
				\textbf{At the start ignoring somethings are fine.}
			\item 
				Make schematic sketch.
			
		\end{enumerate}
	\item planing the approach to solve the problem.
		\begin{enumerate}
			\item make a hypothesis.
			\item Do all the maths and physics and come to some equation on which we can do calculation and comparer that with our experiment that we would not done at that time. 
		\end{enumerate}
	\item experimental setup designing.
		\begin{enumerate}
			\item complete layout of the experiment like a engineer and proper explanation of how we collect the data.( theoretically).
			\item material and .... things require for the experiments.
			\item Do the experiment and take data.
			\item analyse the data.
			\item ....
		\end{enumerate}
	\item making presentation. 
		\begin{enumerate}
			\item make PPT.
			\item make presentation.
			\item practise.
		\end{enumerate}
		


		

\end{enumerate}

\end{document}
  
